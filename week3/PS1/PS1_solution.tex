\documentclass{article}
\usepackage{enumerate}
\usepackage{listings}
\usepackage{verbatim}
\usepackage{amsmath}
\begin{document}

\title{Solution to Problem Set 1}
\author{S Krishna Savant \\ 100070056}
\maketitle

\section{Question 1}
\begin{comment}
1. Convert the following sentences into predicate calculas:
If it does not rain on Monday, Ram will go to the mountains
Cute is a good cat.
All basketball players are tall.
Some people like anchovies.
If wishes are horses, beggars would ride.
Nobody likes Delhi.
\end{comment}

\begin{itemize}
    \item $\neg (\text{Rain on monday}) \implies (\text{Ram goes to mountains})$
    \item \text{(good cat)-Cute}
    \item $ \forall \text{ Basketball Players } \implies  \text{Tall}$
    \item $\exists$ x $\in$ People such that x like anchovies
    \item (Wishes = Horses) $\implies$ beggars ride.
    \item $\neg$ ($\exists$ x such that x like Delhi)
\end{itemize}

\section{Question 2}
\begin{comment}
Attempt to unify the following pairs of expressions either with their most general unifiers or explain
why they will not unify:
p(X,Y) and p(a,Z)
p(X,X) and p(a,b)
Ancestor(X,Y) and Ancestor(x,Father(x))
Ancestor(X,Father(X)) and Ancestor(Ram,Sita)
q(X) and ¬q(a)
\end{comment}

\begin{itemize}
    \item p(X,Y) p(a,Z)
\end{itemize}
\end{document}
